%%=============================================================================
%% Inleiding
% De inleiding moet de lezer net genoeg informatie verschaffen om het onderwerp te begrijpen en in te zien waarom de onderzoeksvraag de moeite waard is om te onderzoeken. In de inleiding ga je literatuurverwijzingen beperken, zodat de tekst vlot leesbaar blijft. Je kan de inleiding verder onderverdelen in secties als dit de tekst verduidelijkt. Zaken die aan bod kunnen komen in de inleiding~\autocite{Pollefliet2011}:
% Uit je probleemstelling moet duidelijk zijn dat je onderzoek een meerwaarde heeft voor een concrete doelgroep. De doelgroep moet goed gedefinieerd en afgelijnd zijn. Doelgroepen als ``bedrijven,'' ``KMO's'', systeembeheerders, enz.~zijn nog te vaag. Als je een lijstje kan maken van de personen/organisaties die een meerwaarde zullen vinden in deze bachelorproef (dit is eigenlijk je steekproefkader), dan is dat een indicatie dat de doelgroep goed gedefinieerd is. Dit kan een enkel bedrijf zijn of zelfs één persoon (je co-promotor/opdrachtgever).
% Wees zo concreet mogelijk bij het formuleren van je onderzoeksvraag. Een onderzoeksvraag is trouwens iets waar nog niemand op dit moment een antwoord heeft (voor zover je kan nagaan). Het opzoeken van bestaande informatie (bv. ``welke tools bestaan er voor deze toepassing?'') is dus geen onderzoeksvraag. Je kan de onderzoeksvraag verder specifiëren in deelvragen. Bv.~als je onderzoek gaat over performantiemetingen, dan 
% Wat is het beoogde resultaat van je graduaatsproef? Wat zijn de criteria voor succes? Beschrijf die zo concreet mogelijk. Gaat het bv.\ om een proof-of-concept, een prototype, een verslag met aanbevelingen, een vergelijkende studie, enz.

% Het is gebruikelijk aan het einde van de inleiding een overzicht te
% geven van de opbouw van de rest van de tekst. Deze sectie bevat al een aanzet
% die je kan aanvullen/aanpassen in functie van je eigen tekst.


% TODO: Vul hier aan voor je eigen hoofstukken, één of twee zinnen per hoofdstuk

%%=============================================================================

\chapter{\IfLanguageName{dutch}{Inleiding}{Introduction}}%
\label{ch:inleiding}


C++ is een taal dat ontstond in de jaren '80 en is een uitbreiding van de programmeertaal C.
Zelfs als het al 40 jaar oud is, is het nog steeds een van de meest populaire programmeertalen ter wereld.
Het wordt vaak gebruikt voor systeemsoftware, game-ontwikkeling en toepassingen die hoge prestaties vereisen. De geschiedenis van informatica 
boeit me, vooral de evolutie van programmeertalen. Omdat C++ zo lang bestaat, zijn er veel verschillende versies van de taal. 
De concepten en technieken die in C++ worden gebruikt, bestaan er voor een reden.
C++ biedt een combinatie van lage-niveau geheugenbeheer en hoge-niveau abstractie, waardoor het een veelzijdige taal is voor verschillende soorten 
softwareontwikkeling.


\section{\IfLanguageName{dutch}{Wat is c++?}{Wat is c++?}}%
\label{sec:Wat is c++?}


In de jaren 70 werden er veel programmeertalen ontwikkeld, hieronder C, Pascal en Fortran voorbeelden hiervan.
Deze talen hadden veel limieten, zoals het gebrek aan objectgeoriënteerd programmeren en het gebrek aan lage-niveau geheugenbeheer.
Bjarne Stroustrup, een Deense computerwetenschapper, begon in 1979 met het ontwikkelen van C++ als een uitbreiding van de programmeertaal C.
C++ is een programmeertaal die de kracht van C combineert met de mogelijkheden van objectgeoriënteerd programmeren.


Het is een veelzijdige taal die vandaag de dag gebruikt wordt voor verschillende soorten softwareontwikkeling, zoals systeemsoftware, game-ontwikkeling en toepassingen die hoge prestaties vereisen.
De ontwikkelaar van de taal haalde inspiratie uit verschillende programmeertalen, zoals Simula, C, Algol 68 en andere talen om C++ te ontwikkelen.
Simula was de pionier op het gebied van objectgeoriënteerd programmeren en introduceerde concepten zoals klassen en objecten.
Het was een taal die kind was van de taal Algol 60, die op zijn beurt weer een invloed had op de ontwikkeling van C.
Later werd Angol 60 herzien en werd het Angol 68.
Terwijl Angol 68 een taal was die invloed heeft gehad op het structureren van code, deze taal introduceerde het gebruik van code blokken.


C++ krijgt elke 3 jaar een nieuwe versie, maar meestal gebruiken programmeers de oudere versies van de compiler.
Er bestaaan ook meerdere verschillende IDEs die C++ ondersteunen, 
zoals Microsoft Visual Studio, Code::Blocks en CLion.  

\section{\IfLanguageName{dutch}{Wat is SFML?}{What is SFML?}}%
\label{sec:Wat is SFML?}

SFML (Simple and Fast Multimedia Library) is een cross-platform software development library ontworpen om een eenvoudige API te bieden voor het programmeren van multimedia applicaties. 
Het is geschreven in C++ en biedt bindingen voor andere talen zoals C, .NET, Python, Java, en meer.
SFML biedt modules voor 2D graphics, audio, windowing, en netwerkfunctionaliteit. 
Het wordt vaak gebruikt voor het ontwikkelen van 2D games en andere interactieve grafische applicaties.


\section{\IfLanguageName{dutch}{Toelichting van mijn keuze}{Research question}}%
\label{sec:Toelichting van mijn keuze}

Ik heb ervoor gekozen om C++ te leren voor mijn graduaatsproef omdat er veel concepten zijn die mij interessant lijken. 
Deze concepten kunnen mij ook helpen bij het schrijven van software in andere programmeertalen. Memory management kennen helpt bij het begrijpen 
van hoe variabelen in het geheugen worden opgeslagen en hoe ze worden beheerd. Pointers en references zijn essentiële concepten 
die ervoor zorgen dat er geen onnogidige kopieën van gegevens worden gemaakt, wat de prestaties van een programma kan verbeteren. 
Dynamische allocatie kan bijleren hoe gegeugen kan gefragmenteerd worden en hoe je dit kan voorkomen. 
Wanneer ik over C++ lees, zie ik vaak dat het een heel krachtige taal is, maar dat het ook complex kan zijn. 
Tijdens mijn graduaatsproef wil ik zelf ervaren hoe het voelt om met C++ te werken en de kracht van de taal te ontdekken. 
Een andere motivatie is dat ik mijn studie wil verder zetten naar een bacheloropleiding. Een mogelijke taal die ik daar kan leren is C++, 

Daarnaast leek SFML interessant voor mij. Het zorgt ervoor dat ik de taal kan leren kennen in
omgeving die creatief is en waar ik mijn eigen ideeën kan uitwerken. 
Er bestaat ook een mogelijkheid om tijdens de zomer verder te werken aan het project, wat mij de kans geeft om meer bij te leren.

\section{\IfLanguageName{dutch}{Onderzoeksdoelstelling}{Research objective}}%
\label{sec:onderzoeksdoelstelling}
Het doel van deze graduaatsproef is C++ te leren kennen en de verschillen tussen 
C++ en C\#, dat wij tijdens deze opleiding hebben geleerd, te ontdekken. 

\section{\IfLanguageName{dutch}{Opzet van deze graduaatsproef}{Structure of this associate thesis}}%
\label{sec:opzet-graduaatsproef}

De rest van deze graduaatsproef is als volgt opgebouwd:

In Hoofdstuk~\ref{ch:stand-van-zaken} wordt een overzicht gegeven van de stand van zaken binnen het onderzoeksdomein, op basis van een literatuurstudie.

In Hoofdstuk~\ref{ch:methodologie} wordt de methodologie toegelicht en worden de gebruikte onderzoekstechnieken besproken om een antwoord te kunnen formuleren op de onderzoeksvragen.


In Hoofdstuk~\ref{ch:conclusie}, tenslotte, wordt de conclusie gegeven en een antwoord geformuleerd op de onderzoeksvragen. Daarbij wordt ook een aanzet gegeven voor toekomstig onderzoek binnen dit domein.