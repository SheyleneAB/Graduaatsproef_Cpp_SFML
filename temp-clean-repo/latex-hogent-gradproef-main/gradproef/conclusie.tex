%%=============================================================================
%% Conclusie
%%=============================================================================

\chapter{Conclusie}%
\label{ch:conclusie}

% TODO: Trek een duidelijke conclusie, in de vorm van een antwoord op de
% onderzoeksvra(a)g(en). Wat was jouw bijdrage aan het onderzoeksdomein en
% hoe biedt dit meerwaarde aan het vakgebied/doelgroep? 
% Reflecteer kritisch over het resultaat. In Engelse teksten wordt deze sectie
% ``Discussion'' genoemd. Had je deze uitkomst verwacht? Zijn er zaken die nog
% niet duidelijk zijn?
% Heeft het onderzoek geleid tot nieuwe vragen die uitnodigen tot verder 
%onderzoek?

Het beginnnen aan mijn project was niet makkelijk. 
Als ik iets van ervaring had, dan zou ik meteen gestart kunnen zijn met het maken van de game.
Uitdagingen worden vaak verwelkomt door mij en ik had wel plezier tijdens het studeren.
De onderzoek naar de geschiedenis boeide mij enorm. 
Ik heb geleerd hoe je moet doorzetten terwijl je vast zit.
Twijfelen gebeurde veel, maar ik heb ervoor gezorgd dat ik niet opgaf.
\\

Ik kan toegeven dat ik liever een project had gemaakt zonder Unreal Engine, met enkel C++.
Het visuele gedeelte van Unreal Engine was zenuwverwekkend door de tijdsgebrek.
Door de stage was het moeilijk om tijd te vinden om aan het project te werken.
SFML is een bibliotheek dat ik heel snel gekozen had, omdat het een eenvoudige API biedt voor het programmeren van multimedia applicaties.
Ik wou meer focussen op de taal zelf en niet op de engine. Lieft had ik vanaf het begin zo gewerkt, maar ik heb er veel van geleerd.
Omdat ik zo weinig ervaring had met Unreal Engine, leek het mogelijk te zijn om veel code te schrijven hiervoor.
Dat zou wel misschien mogelijk geweest zijn, maar door de obstakels die ik tegenkwam, was het moeilijk om verder te gaan.
Daarom was het een opluchting dat ik snel kon overstappen naar SFML.

De taal zelf leren was een gedeelte waarvan ik genoot en in de toekomst wil ik er zeker mee verder gaan.




