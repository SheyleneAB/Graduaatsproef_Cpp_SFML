%%=============================================================================
%% Methodologie
%%=============================================================================

\chapter{\IfLanguageName{dutch}{Methodologie}{Methodology}}%
\label{ch:methodologie}

%% TODO: In dit hoofstuk geef je een korte toelichting over hoe je te werk bent
%% gegaan. Verdeel je onderzoek in grote fasen, en licht in elke fase toe wat
%% de doelstelling was, welke deliverables daar uit gekomen zijn, en welke
%% onderzoeksmethoden je daarbij toegepast hebt. Verantwoord waarom je
%% op deze manier te werk gegaan bent.
%% 
%% Voorbeelden van zulke fasen zijn: literatuurstudie, opstellen van een
%% requirements-analyse, opstellen long-list (bij vergelijkende studie),
%% selectie van geschikte tools (bij vergelijkende studie, "short-list"),
%% opzetten testopstelling/PoC, uitvoeren testen en verzamelen
%% van resultaten, analyse van resultaten, ...
%%
%% !!!!! LET OP !!!!!
%%
%% Het is uitdrukkelijk NIET de bedoeling dat je het grootste deel van de corpus
%% van je graduaatsproef in dit hoofstuk verwerkt! Dit hoofdstuk is eerder een
%% kort overzicht van je plan van aanpak.
%%
%% Maak voor elke fase (behalve het literatuuronderzoek) een NIEUW HOOFDSTUK aan
%% en geef het een gepaste titel.

De graduaatsproef begon met een literatuurstudie over C++ in het algemeen. Er werd opgezocht hoe de verschillende concepten van de taal werken.
Verschillende bronnen werden geraadpleegd, zoals boeken, online tutorials en documentatie van de taal zelf. 
Sommige bronnen waren meer gericht op beginners, terwijl andere meer diepgaand waren en zich richtten op gevorderde gebruikers.
Op het einde van de studie werden de fundamenten gelegd voor het project zodat ze toegepast konden worden.
\\

De volgende stap was experimenteren met de taal zelf in Microsoft Visual Studio.
Een paar kleine projecten werden gemaakt om de basisconcepten van de taal te begrijpen, zoals variabelen, loops, functies en classes.
Deze projecten hielpen met de syntax en mechanismen van de taal te begrijpen. 
Tijdens de finale versie met Unreal Engine 5 zou er dan teruggevallen kunnen worden op deze kennis.
\\

Tenslotte er dan begonnen met het finale project in Unreal Engine 5.
De opzetting van het project werd gedaan en Unreal Engine werd bestudeerd.
Een Proof of Concept werd gemaakt als een finale deel van de graduaatsproef.

