%%=============================================================================
%% Samenvatting
%%=============================================================================

% TODO: De "abstract" of samenvatting is een kernachtige (~ 1 blz. voor een
% thesis) synthese van het document.
%
% Een goede abstract biedt een kernachtig antwoord op volgende vragen:
%
% 1. Waarover gaat de graduaatsproef?
% 2. Waarom heb je er over geschreven?
% 3. Hoe heb je het onderzoek uitgevoerd?
% 4. Wat waren de resultaten? Wat blijkt uit je onderzoek?
% 5. Wat betekenen je resultaten? Wat is de relevantie voor het werkveld?
%
% Daarom bestaat een abstract uit volgende componenten:
%
% - inleiding + kaderen thema
% - probleemstelling
% - (centrale) onderzoeksvraag
% - onderzoeksdoelstelling
% - methodologie
% - resultaten (beperk tot de belangrijkste, relevant voor de onderzoeksvraag)
% - conclusies, aanbevelingen, beperkingen
%
% LET OP! Een samenvatting is GEEN voorwoord!

%%---------- Nederlandse samenvatting -----------------------------------------
%
% TODO: Als je je graduaatsproef in het Engels schrijft, moet je eerst een
% Nederlandse samenvatting invoegen. Haal daarvoor onderstaande code uit
% commentaar.
% Wie zijn/haar graduaatsproef in het Nederlands schrijft, kan dit negeren, de inhoud
% wordt niet in het document ingevoegd.

\IfLanguageName{dutch}{
\selectlanguage{dutch}
\chapter*{Samenvatting}

Op het einde van onze opleiding hebben we de opdracht gekregen om een nieuwe programmeertaal te leren en een project te maken in deze taal. 
We konden kiezen om ofwel bij ons stagebedrijf een project te doen, ofwel een project te doen met een zelfgekozen onderwerp.
\\
\\
Ik heb ervoor gekozen om zelfstandig in C++ een game te maken met Unreal Engine 5. 
Dit project is een grote uitdaging geweest, omdat ik nog nooit eerder met C++ of Unreal Engine 5 had gewerkt.
Sommige concepten van de taal waren mij bekend omdat ik al eerder met C\# had gewerkt, maar de taal zelf was nieuw voor mij.
In het begin was ik begonnen met onderzoek naar de taal en de engine, tegelijkertijd onderzocht ik of er mogelijkheden waren 
de game efficienter te maken met behulp van design patterns.

Later schakelde ik over naar SFML, omdat Unreal Engine 5 niet goed werkte en ik meer wilde focussen op de taal zelf.
}

%%---------- Samenvatting -----------------------------------------------------
% De samenvatting in de hoofdtaal van het document

\chapter*{\IfLanguageName{dutch}{Summary}{Summary}}

At the end of our studies, we were given the assignment to learn a new programming language and create a project using this language.
We could choose either to do a project at our internship company or to work on a self-chosen topic.
\\
\\
I chose to independently create a game in C++ using Unreal Engine 5.
This project was a major challenge, as I had never worked with C++ or Unreal Engine 5 before.
Some concepts of the language were familiar to me because I had previously worked with C\#, but the language itself was new to me.
In the beginning, I started by researching the language and the engine, while also investigating whether there were possibilities to make the game more efficient using design patterns.